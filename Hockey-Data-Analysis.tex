% Options for packages loaded elsewhere
\PassOptionsToPackage{unicode}{hyperref}
\PassOptionsToPackage{hyphens}{url}
\PassOptionsToPackage{dvipsnames,svgnames,x11names}{xcolor}
%
\documentclass[
  a3paper,
]{article}
\usepackage{amsmath,amssymb}
\usepackage{iftex}
\ifPDFTeX
  \usepackage[T1]{fontenc}
  \usepackage[utf8]{inputenc}
  \usepackage{textcomp} % provide euro and other symbols
\else % if luatex or xetex
  \usepackage{unicode-math} % this also loads fontspec
  \defaultfontfeatures{Scale=MatchLowercase}
  \defaultfontfeatures[\rmfamily]{Ligatures=TeX,Scale=1}
\fi
\usepackage{lmodern}
\ifPDFTeX\else
  % xetex/luatex font selection
\fi
% Use upquote if available, for straight quotes in verbatim environments
\IfFileExists{upquote.sty}{\usepackage{upquote}}{}
\IfFileExists{microtype.sty}{% use microtype if available
  \usepackage[]{microtype}
  \UseMicrotypeSet[protrusion]{basicmath} % disable protrusion for tt fonts
}{}
\makeatletter
\@ifundefined{KOMAClassName}{% if non-KOMA class
  \IfFileExists{parskip.sty}{%
    \usepackage{parskip}
  }{% else
    \setlength{\parindent}{0pt}
    \setlength{\parskip}{6pt plus 2pt minus 1pt}}
}{% if KOMA class
  \KOMAoptions{parskip=half}}
\makeatother
\usepackage{xcolor}
\usepackage[margin=1in]{geometry}
\usepackage{color}
\usepackage{fancyvrb}
\newcommand{\VerbBar}{|}
\newcommand{\VERB}{\Verb[commandchars=\\\{\}]}
\DefineVerbatimEnvironment{Highlighting}{Verbatim}{commandchars=\\\{\}}
% Add ',fontsize=\small' for more characters per line
\usepackage{framed}
\definecolor{shadecolor}{RGB}{248,248,248}
\newenvironment{Shaded}{\begin{snugshade}}{\end{snugshade}}
\newcommand{\AlertTok}[1]{\textcolor[rgb]{0.94,0.16,0.16}{#1}}
\newcommand{\AnnotationTok}[1]{\textcolor[rgb]{0.56,0.35,0.01}{\textbf{\textit{#1}}}}
\newcommand{\AttributeTok}[1]{\textcolor[rgb]{0.13,0.29,0.53}{#1}}
\newcommand{\BaseNTok}[1]{\textcolor[rgb]{0.00,0.00,0.81}{#1}}
\newcommand{\BuiltInTok}[1]{#1}
\newcommand{\CharTok}[1]{\textcolor[rgb]{0.31,0.60,0.02}{#1}}
\newcommand{\CommentTok}[1]{\textcolor[rgb]{0.56,0.35,0.01}{\textit{#1}}}
\newcommand{\CommentVarTok}[1]{\textcolor[rgb]{0.56,0.35,0.01}{\textbf{\textit{#1}}}}
\newcommand{\ConstantTok}[1]{\textcolor[rgb]{0.56,0.35,0.01}{#1}}
\newcommand{\ControlFlowTok}[1]{\textcolor[rgb]{0.13,0.29,0.53}{\textbf{#1}}}
\newcommand{\DataTypeTok}[1]{\textcolor[rgb]{0.13,0.29,0.53}{#1}}
\newcommand{\DecValTok}[1]{\textcolor[rgb]{0.00,0.00,0.81}{#1}}
\newcommand{\DocumentationTok}[1]{\textcolor[rgb]{0.56,0.35,0.01}{\textbf{\textit{#1}}}}
\newcommand{\ErrorTok}[1]{\textcolor[rgb]{0.64,0.00,0.00}{\textbf{#1}}}
\newcommand{\ExtensionTok}[1]{#1}
\newcommand{\FloatTok}[1]{\textcolor[rgb]{0.00,0.00,0.81}{#1}}
\newcommand{\FunctionTok}[1]{\textcolor[rgb]{0.13,0.29,0.53}{\textbf{#1}}}
\newcommand{\ImportTok}[1]{#1}
\newcommand{\InformationTok}[1]{\textcolor[rgb]{0.56,0.35,0.01}{\textbf{\textit{#1}}}}
\newcommand{\KeywordTok}[1]{\textcolor[rgb]{0.13,0.29,0.53}{\textbf{#1}}}
\newcommand{\NormalTok}[1]{#1}
\newcommand{\OperatorTok}[1]{\textcolor[rgb]{0.81,0.36,0.00}{\textbf{#1}}}
\newcommand{\OtherTok}[1]{\textcolor[rgb]{0.56,0.35,0.01}{#1}}
\newcommand{\PreprocessorTok}[1]{\textcolor[rgb]{0.56,0.35,0.01}{\textit{#1}}}
\newcommand{\RegionMarkerTok}[1]{#1}
\newcommand{\SpecialCharTok}[1]{\textcolor[rgb]{0.81,0.36,0.00}{\textbf{#1}}}
\newcommand{\SpecialStringTok}[1]{\textcolor[rgb]{0.31,0.60,0.02}{#1}}
\newcommand{\StringTok}[1]{\textcolor[rgb]{0.31,0.60,0.02}{#1}}
\newcommand{\VariableTok}[1]{\textcolor[rgb]{0.00,0.00,0.00}{#1}}
\newcommand{\VerbatimStringTok}[1]{\textcolor[rgb]{0.31,0.60,0.02}{#1}}
\newcommand{\WarningTok}[1]{\textcolor[rgb]{0.56,0.35,0.01}{\textbf{\textit{#1}}}}
\usepackage{graphicx}
\makeatletter
\def\maxwidth{\ifdim\Gin@nat@width>\linewidth\linewidth\else\Gin@nat@width\fi}
\def\maxheight{\ifdim\Gin@nat@height>\textheight\textheight\else\Gin@nat@height\fi}
\makeatother
% Scale images if necessary, so that they will not overflow the page
% margins by default, and it is still possible to overwrite the defaults
% using explicit options in \includegraphics[width, height, ...]{}
\setkeys{Gin}{width=\maxwidth,height=\maxheight,keepaspectratio}
% Set default figure placement to htbp
\makeatletter
\def\fps@figure{htbp}
\makeatother
\setlength{\emergencystretch}{3em} % prevent overfull lines
\providecommand{\tightlist}{%
  \setlength{\itemsep}{0pt}\setlength{\parskip}{0pt}}
\setcounter{secnumdepth}{-\maxdimen} % remove section numbering
\ifLuaTeX
  \usepackage{selnolig}  % disable illegal ligatures
\fi
\IfFileExists{bookmark.sty}{\usepackage{bookmark}}{\usepackage{hyperref}}
\IfFileExists{xurl.sty}{\usepackage{xurl}}{} % add URL line breaks if available
\urlstyle{same}
\hypersetup{
  pdftitle={STAT 847: Final Project},
  colorlinks=true,
  linkcolor={Maroon},
  filecolor={Maroon},
  citecolor={Blue},
  urlcolor={blue},
  pdfcreator={LaTeX via pandoc}}

\title{STAT 847: Final Project}
\usepackage{etoolbox}
\makeatletter
\providecommand{\subtitle}[1]{% add subtitle to \maketitle
  \apptocmd{\@title}{\par {\large #1 \par}}{}{}
}
\makeatother
\subtitle{DUE: Friday April 19, 2024 by 11:59pm Eastern}
\author{}
\date{\vspace{-2.5em}}

\begin{document}
\maketitle

\begin{Shaded}
\begin{Highlighting}[]
\FunctionTok{library}\NormalTok{(plyr)}
\FunctionTok{library}\NormalTok{(ggplot2)}
\FunctionTok{library}\NormalTok{(tidyr)}
\FunctionTok{library}\NormalTok{(dplyr)}
\end{Highlighting}
\end{Shaded}

\begin{verbatim}
## 
## Attaching package: 'dplyr'
\end{verbatim}

\begin{verbatim}
## The following objects are masked from 'package:plyr':
## 
##     arrange, count, desc, failwith, id, mutate, rename, summarise,
##     summarize
\end{verbatim}

\begin{verbatim}
## The following objects are masked from 'package:stats':
## 
##     filter, lag
\end{verbatim}

\begin{verbatim}
## The following objects are masked from 'package:base':
## 
##     intersect, setdiff, setequal, union
\end{verbatim}

\begin{Shaded}
\begin{Highlighting}[]
\FunctionTok{library}\NormalTok{(readr)}
\FunctionTok{library}\NormalTok{(GGally)}
\end{Highlighting}
\end{Shaded}

\begin{verbatim}
## Registered S3 method overwritten by 'GGally':
##   method from   
##   +.gg   ggplot2
\end{verbatim}

\begin{Shaded}
\begin{Highlighting}[]
\FunctionTok{library}\NormalTok{(rpart)}
\FunctionTok{library}\NormalTok{(rpart.plot)}
\FunctionTok{library}\NormalTok{(FactoMineR)}
\FunctionTok{library}\NormalTok{(MASS)}
\end{Highlighting}
\end{Shaded}

\begin{verbatim}
## 
## Attaching package: 'MASS'
\end{verbatim}

\begin{verbatim}
## The following object is masked from 'package:dplyr':
## 
##     select
\end{verbatim}

\begin{Shaded}
\begin{Highlighting}[]
\FunctionTok{library}\NormalTok{(rsample)}
\FunctionTok{library}\NormalTok{(rattle)}
\end{Highlighting}
\end{Shaded}

\begin{verbatim}
## Loading required package: tibble
\end{verbatim}

\begin{verbatim}
## Loading required package: bitops
\end{verbatim}

\begin{verbatim}
## Rattle: A free graphical interface for data science with R.
## Version 5.5.1 Copyright (c) 2006-2021 Togaware Pty Ltd.
## Type 'rattle()' to shake, rattle, and roll your data.
\end{verbatim}

\begin{Shaded}
\begin{Highlighting}[]
\FunctionTok{library}\NormalTok{(RColorBrewer)}
\FunctionTok{library}\NormalTok{(lubridate)}
\end{Highlighting}
\end{Shaded}

\begin{verbatim}
## 
## Attaching package: 'lubridate'
\end{verbatim}

\begin{verbatim}
## The following objects are masked from 'package:base':
## 
##     date, intersect, setdiff, union
\end{verbatim}

\begin{Shaded}
\begin{Highlighting}[]
\FunctionTok{library}\NormalTok{(leaps)}
\FunctionTok{library}\NormalTok{(png)}
\FunctionTok{library}\NormalTok{(grid)}
\FunctionTok{library}\NormalTok{(gridExtra)}
\end{Highlighting}
\end{Shaded}

\begin{verbatim}
## 
## Attaching package: 'gridExtra'
\end{verbatim}

\begin{verbatim}
## The following object is masked from 'package:dplyr':
## 
##     combine
\end{verbatim}

\begin{Shaded}
\begin{Highlighting}[]
\FunctionTok{library}\NormalTok{(scales)}
\end{Highlighting}
\end{Shaded}

\begin{verbatim}
## 
## Attaching package: 'scales'
\end{verbatim}

\begin{verbatim}
## The following object is masked from 'package:readr':
## 
##     col_factor
\end{verbatim}

\begin{Shaded}
\begin{Highlighting}[]
\FunctionTok{library}\NormalTok{(}\StringTok{"factoextra"}\NormalTok{)}
\end{Highlighting}
\end{Shaded}

\begin{verbatim}
## Welcome! Want to learn more? See two factoextra-related books at https://goo.gl/ve3WBa
\end{verbatim}

\begin{Shaded}
\begin{Highlighting}[]
\NormalTok{dat }\OtherTok{=} \FunctionTok{read.csv}\NormalTok{(}\StringTok{"./Big{-}Data{-}Cup{-}2024{-}main/BDC\_2024\_Womens\_Data.csv"}\NormalTok{)}
\NormalTok{dat}\SpecialCharTok{$}\NormalTok{Period }\OtherTok{\textless{}{-}} \FunctionTok{as.character}\NormalTok{(dat}\SpecialCharTok{$}\NormalTok{Period)}
\end{Highlighting}
\end{Shaded}

\newpage

\vspace{2cm}

\begin{enumerate}
\def\labelenumi{\arabic{enumi})}
\tightlist
\item
  \textbf{MUST BE INCLUDED} Describe and justify two different topics or
  approaches you might want to consider for this dataset and task. You
  don't have to use these tasks in the actual analysis.
\end{enumerate}

In this project, I'm aiming to use the following techniques to do the
analysis:\\

Regression Analysis: Use regression to predict game outcomes.
player/team performance, game conditions based on the information
provided in the data set, such as events, goals, clock, period and
details.\\

Classification Algorithms: Use logistic regression, decision trees,
random forests, classification trees to classify game outcomes. We can
also classify event based on the coordinates informations.\\

Clustering: Use K-means clustering can help identify similar players
based on their performance.\\

Time Series Analysis: As in the data set there're some columns includes
time aspects, Time series techniques can be helpful to analyze trends
and make predictions. ~

Two major topics:\\

Topic 1: Individual player performance In this exploration of individual
player performance, we delve into the intricate details of each player's
contributions throughout the game. The analysis needs to focus on every
event in which they participated. We need to focus on the frequency of
each player's involvement in various event types, ranging from shots and
plays to takeaways, puck recoveries, dump-ins, dump-outs, zone entries,
faceoffs, and penalties. By dissecting their participation in these
events, we gain information about how much they contribute to the game.
Also, we can develop performance metrics for each event type to
accurately assess each player's impact. Metrics can be shoting
efficiency, which calculates the ratio of successful shots to total
shots, providing a measure of offensive prowess. Also, we calculate
defensive Impact by evaluating the frequency of takeaways and puck
recoveries, shedding light on a player's defensive prowess and ability
to disrupt the opponent's game plan. Through this meticulous analysis
and the development of specific metrics, we aim to provide a
comprehensive evaluation of each player's performance. By quantifying
their contributions across various facets of the game, we gain deeper
insights into their strengths, weaknesses, and overall impact on the
team's performance. Furthermore, to improve our analysis, we can
generate visual charts.\\

To approach Topic 1, we will need to make some modifications to the data
set to gather information from it, So the data set will focus more on
players' performance but not the each event, We will first generate some
subsets which will group the data set by Players, Event, Date, Period,
calculate the number of each events that each players involved in and
average coordinates of this player when event happens:

\begin{Shaded}
\begin{Highlighting}[]
\NormalTok{player\_dat }\OtherTok{\textless{}{-}}\NormalTok{ dat }\SpecialCharTok{\%\textgreater{}\%} \FunctionTok{group\_by}\NormalTok{(Player, Date, Event, Team, Period) }\SpecialCharTok{\%\textgreater{}\%} 
  \FunctionTok{summarise}\NormalTok{(}\AttributeTok{avg\_x =} \FunctionTok{mean}\NormalTok{(X.Coordinate), }\AttributeTok{avg\_y =} \FunctionTok{mean}\NormalTok{(Y.Coordinate), }\AttributeTok{num\_event =} \FunctionTok{n}\NormalTok{())}
\end{Highlighting}
\end{Shaded}

\begin{verbatim}
## `summarise()` has grouped output by 'Player', 'Date', 'Event', 'Team'. You can
## override using the `.groups` argument.
\end{verbatim}

\begin{Shaded}
\begin{Highlighting}[]
\CommentTok{\# head(player\_dat)}

\CommentTok{\# regular time}
\NormalTok{player\_dat\_reg }\OtherTok{\textless{}{-}} \FunctionTok{filter}\NormalTok{(player\_dat, Period }\SpecialCharTok{!=} \DecValTok{4}\NormalTok{)}


\CommentTok{\# specific Offense Performance}

\NormalTok{player\_dat\_off }\OtherTok{\textless{}{-}}\NormalTok{  dat }\SpecialCharTok{\%\textgreater{}\%}
  \FunctionTok{filter}\NormalTok{(Event }\SpecialCharTok{\%in\%} \FunctionTok{c}\NormalTok{(}\StringTok{"Goal"}\NormalTok{, }\StringTok{"Incomplete Play"}\NormalTok{, }\StringTok{"Play"}\NormalTok{, }\StringTok{"Shot"}\NormalTok{, }\StringTok{"Zone Entry"}\NormalTok{))}\SpecialCharTok{\%\textgreater{}\%}
  \FunctionTok{group\_by}\NormalTok{(Player, Team, Date, Event, Period, X.Coordinate, Y.Coordinate) }\SpecialCharTok{\%\textgreater{}\%}
  \FunctionTok{summarise}\NormalTok{(}\AttributeTok{num\_event =} \FunctionTok{n}\NormalTok{())}
\end{Highlighting}
\end{Shaded}

\begin{verbatim}
## `summarise()` has grouped output by 'Player', 'Team', 'Date', 'Event',
## 'Period', 'X.Coordinate'. You can override using the `.groups` argument.
\end{verbatim}

\begin{Shaded}
\begin{Highlighting}[]
\CommentTok{\# head(player\_dat\_off)}


\CommentTok{\# specific shoting Performance}

\NormalTok{player\_dat\_shot }\OtherTok{\textless{}{-}}\NormalTok{  dat }\SpecialCharTok{\%\textgreater{}\%}
  \FunctionTok{filter}\NormalTok{(Event }\SpecialCharTok{\%in\%} \FunctionTok{c}\NormalTok{(}\StringTok{"Goal"}\NormalTok{, }\StringTok{"Shot"}\NormalTok{))}\SpecialCharTok{\%\textgreater{}\%} 
  \FunctionTok{group\_by}\NormalTok{(Player, Team, Date, Event, Period, X.Coordinate, Y.Coordinate) }\SpecialCharTok{\%\textgreater{}\%}
  \FunctionTok{summarise}\NormalTok{(}\AttributeTok{num\_event =} \FunctionTok{n}\NormalTok{())}
\end{Highlighting}
\end{Shaded}

\begin{verbatim}
## `summarise()` has grouped output by 'Player', 'Team', 'Date', 'Event',
## 'Period', 'X.Coordinate'. You can override using the `.groups` argument.
\end{verbatim}

\begin{Shaded}
\begin{Highlighting}[]
\CommentTok{\# head(player\_dat\_shot)}


\CommentTok{\# specific Passing Performance}

\NormalTok{player\_dat\_pass }\OtherTok{\textless{}{-}}\NormalTok{  dat }\SpecialCharTok{\%\textgreater{}\%}
  \FunctionTok{filter}\NormalTok{(Event }\SpecialCharTok{\%in\%} \FunctionTok{c}\NormalTok{(}\StringTok{"Incomplete Play"}\NormalTok{, }\StringTok{"Play"}\NormalTok{))}\SpecialCharTok{\%\textgreater{}\%} 
  \FunctionTok{group\_by}\NormalTok{(Player, Team, Date, Event, Period, X.Coordinate, Y.Coordinate) }\SpecialCharTok{\%\textgreater{}\%}
  \FunctionTok{summarise}\NormalTok{(}\AttributeTok{num\_event =} \FunctionTok{n}\NormalTok{())}
\end{Highlighting}
\end{Shaded}

\begin{verbatim}
## `summarise()` has grouped output by 'Player', 'Team', 'Date', 'Event',
## 'Period', 'X.Coordinate'. You can override using the `.groups` argument.
\end{verbatim}

\begin{Shaded}
\begin{Highlighting}[]
\CommentTok{\# head(player\_dat\_pass)}


\CommentTok{\# Defense Performance}

\NormalTok{player\_dat\_def }\OtherTok{\textless{}{-}}\NormalTok{ dat }\SpecialCharTok{\%\textgreater{}\%} 
  \FunctionTok{anti\_join}\NormalTok{(player\_dat\_off)}\SpecialCharTok{\%\textgreater{}\%} 
  \FunctionTok{group\_by}\NormalTok{(Player, Team, Date, Event, Period, X.Coordinate, Y.Coordinate) }\SpecialCharTok{\%\textgreater{}\%}
  \FunctionTok{summarise}\NormalTok{(}\AttributeTok{num\_event =} \FunctionTok{n}\NormalTok{())}
\end{Highlighting}
\end{Shaded}

\begin{verbatim}
## Joining with `by = join_by(Date, Period, Team, Player, Event, X.Coordinate,
## Y.Coordinate)`
## `summarise()` has grouped output by 'Player', 'Team', 'Date', 'Event',
## 'Period', 'X.Coordinate'. You can override using the `.groups` argument.
\end{verbatim}

\begin{Shaded}
\begin{Highlighting}[]
\CommentTok{\# head(player\_dat\_def)}
\end{Highlighting}
\end{Shaded}

Task 2: Team strategy In this topic, we will focus on analyzing team
performance involving different aspects of the game, including offensive
and defensive actions, according to the dataset, the analysis is based
on: We will start by tracking the number of shots attempted by each
team, focusing on shots on goal, missed shots, and blocked shots. This
data will provide valuable insights into the team's offensive strategies
and their ability to create scoring opportunities. Additionally,
analyzing shot locations will allow us to identify high-danger areas
where the team is most effective at generating scoring chances. Moving
on to passing, we will monitor successful pass attempts and puck
movement to assess the team's ability to maintain possession and
orchestrate offensive plays. By examining the types of passes and their
accuracy, we can gain a deeper understanding of the team's passing
efficiency and playmaking capabilities. In terms of puck recoveries, we
will evaluate how quickly the team regains possession after a turnover.
This metric will shed light on the team's resilience and determination
in winning back the puck, as well as their ability to transition from
defence to offence effectively. Furthermore, we will keep track of
takeaways to measure the team's defensive pressure and their ability to
disrupt the opponent's plays.\\

To approach Topic 2, we will need to make some modifications to the data
set to gather information from it, So the data set will focus more on
Teams' performance but not the each event, We will first generate some
subsets which will group the data set by Teams (Home/Away/Team), Event,
Date, Period, calculate the number of each events that each Team
involved in and average coordinates of this player when event happens:

\begin{Shaded}
\begin{Highlighting}[]
\NormalTok{team\_dat }\OtherTok{\textless{}{-}}\NormalTok{ dat }\SpecialCharTok{\%\textgreater{}\%} \FunctionTok{group\_by}\NormalTok{(Team, Date, Event, Period) }\SpecialCharTok{\%\textgreater{}\%} \FunctionTok{summarise}\NormalTok{(}\AttributeTok{avg\_x =} \FunctionTok{mean}\NormalTok{(X.Coordinate), }\AttributeTok{avg\_y =} \FunctionTok{mean}\NormalTok{(Y.Coordinate), }\AttributeTok{num\_event =} \FunctionTok{n}\NormalTok{())}
\end{Highlighting}
\end{Shaded}

\begin{verbatim}
## `summarise()` has grouped output by 'Team', 'Date', 'Event'. You can override
## using the `.groups` argument.
\end{verbatim}

\begin{Shaded}
\begin{Highlighting}[]
\CommentTok{\# regular time overall}
\NormalTok{team\_dat\_reg }\OtherTok{\textless{}{-}} \FunctionTok{filter}\NormalTok{(team\_dat, Period }\SpecialCharTok{!=} \DecValTok{4}\NormalTok{)}


\CommentTok{\# specific Offense Performance}

\NormalTok{team\_dat\_off }\OtherTok{\textless{}{-}}\NormalTok{  dat }\SpecialCharTok{\%\textgreater{}\%}
  \FunctionTok{filter}\NormalTok{(Event }\SpecialCharTok{\%in\%} \FunctionTok{c}\NormalTok{(}\StringTok{"Goal"}\NormalTok{, }\StringTok{"Incomplete Play"}\NormalTok{, }\StringTok{"Play"}\NormalTok{, }\StringTok{"Shot"}\NormalTok{, }\StringTok{"Zone Entry"}\NormalTok{))}\SpecialCharTok{\%\textgreater{}\%}
  \FunctionTok{group\_by}\NormalTok{(Team, Date, Event, Period, X.Coordinate, Y.Coordinate) }\SpecialCharTok{\%\textgreater{}\%} 
  \FunctionTok{summarise}\NormalTok{(}\AttributeTok{num\_event =} \FunctionTok{n}\NormalTok{())}
\end{Highlighting}
\end{Shaded}

\begin{verbatim}
## `summarise()` has grouped output by 'Team', 'Date', 'Event', 'Period',
## 'X.Coordinate'. You can override using the `.groups` argument.
\end{verbatim}

\begin{Shaded}
\begin{Highlighting}[]
\CommentTok{\# head(team\_dat\_off)}

\CommentTok{\# specific shoting Performance}

\NormalTok{team\_dat\_shot }\OtherTok{\textless{}{-}}\NormalTok{  dat }\SpecialCharTok{\%\textgreater{}\%}
  \FunctionTok{filter}\NormalTok{(Event }\SpecialCharTok{\%in\%} \FunctionTok{c}\NormalTok{(}\StringTok{"Goal"}\NormalTok{, }\StringTok{"Shot"}\NormalTok{))}\SpecialCharTok{\%\textgreater{}\%} 
  \FunctionTok{group\_by}\NormalTok{(Team, Date, Event, Period, X.Coordinate, Y.Coordinate) }\SpecialCharTok{\%\textgreater{}\%} 
  \FunctionTok{summarise}\NormalTok{(}\AttributeTok{num\_event =} \FunctionTok{n}\NormalTok{())}
\end{Highlighting}
\end{Shaded}

\begin{verbatim}
## `summarise()` has grouped output by 'Team', 'Date', 'Event', 'Period',
## 'X.Coordinate'. You can override using the `.groups` argument.
\end{verbatim}

\begin{Shaded}
\begin{Highlighting}[]
\CommentTok{\# head(team\_dat\_shot)}


\CommentTok{\# specific Passing Performance}

\NormalTok{team\_dat\_pass }\OtherTok{\textless{}{-}}\NormalTok{  dat }\SpecialCharTok{\%\textgreater{}\%}
  \FunctionTok{filter}\NormalTok{(Event }\SpecialCharTok{\%in\%} \FunctionTok{c}\NormalTok{(}\StringTok{"Incomplete Play"}\NormalTok{, }\StringTok{"Play"}\NormalTok{))}\SpecialCharTok{\%\textgreater{}\%}
  \FunctionTok{group\_by}\NormalTok{(Team, Date, Event, Period, X.Coordinate, Y.Coordinate) }\SpecialCharTok{\%\textgreater{}\%} 
  \FunctionTok{summarise}\NormalTok{(}\AttributeTok{num\_event =} \FunctionTok{n}\NormalTok{())}
\end{Highlighting}
\end{Shaded}

\begin{verbatim}
## `summarise()` has grouped output by 'Team', 'Date', 'Event', 'Period',
## 'X.Coordinate'. You can override using the `.groups` argument.
\end{verbatim}

\begin{Shaded}
\begin{Highlighting}[]
\CommentTok{\# head(team\_dat\_pass)}



\CommentTok{\# Defense Performance}

\NormalTok{team\_dat\_def }\OtherTok{\textless{}{-}}\NormalTok{ dat }\SpecialCharTok{\%\textgreater{}\%} 
  \FunctionTok{anti\_join}\NormalTok{(team\_dat\_off)}\SpecialCharTok{\%\textgreater{}\%} \FunctionTok{group\_by}\NormalTok{(Team, Date, Event, Period, X.Coordinate, Y.Coordinate) }\SpecialCharTok{\%\textgreater{}\%} \FunctionTok{summarise}\NormalTok{(}\AttributeTok{num\_event =} \FunctionTok{n}\NormalTok{())}
\end{Highlighting}
\end{Shaded}

\begin{verbatim}
## Joining with `by = join_by(Date, Period, Team, Event, X.Coordinate,
## Y.Coordinate)`
## `summarise()` has grouped output by 'Team', 'Date', 'Event', 'Period',
## 'X.Coordinate'. You can override using the `.groups` argument.
\end{verbatim}

The subsets computed above may used along with the main data set to do
the analysis.

\newpage

\vspace{2cm}

\begin{enumerate}
\def\labelenumi{\arabic{enumi})}
\setcounter{enumi}{1}
\tightlist
\item
  \textbf{MUST BE INCLUDED} Give a ggpairs plot of what you think are
  the six most important variables. At least one must be categorical,
  and one continuous. Explain your choice of variables and the trends
  between them.
\end{enumerate}

\begin{Shaded}
\begin{Highlighting}[]
\NormalTok{pm }\OtherTok{\textless{}{-}} \FunctionTok{ggpairs}\NormalTok{(team\_dat\_shot, }\AttributeTok{columns =} \FunctionTok{c}\NormalTok{(}\StringTok{"Period"}\NormalTok{, }\StringTok{"Date"}\NormalTok{, }\StringTok{"X.Coordinate"}\NormalTok{, }\StringTok{"Y.Coordinate"}\NormalTok{,}\StringTok{"Event"}\NormalTok{),  }\AttributeTok{mapping=}\FunctionTok{aes}\NormalTok{(}\AttributeTok{color =}\NormalTok{ Team, }\AttributeTok{alpha=}\FloatTok{0.1}\NormalTok{),}
              \AttributeTok{cardinality\_threshold =} \ConstantTok{NULL}\NormalTok{,}
  \AttributeTok{title =} \StringTok{"BDC 2024 Womens shoting Performance"}\NormalTok{)}
\NormalTok{pm}
\end{Highlighting}
\end{Shaded}

\begin{verbatim}
## `stat_bin()` using `bins = 30`. Pick better value with `binwidth`.
## `stat_bin()` using `bins = 30`. Pick better value with `binwidth`.
## `stat_bin()` using `bins = 30`. Pick better value with `binwidth`.
## `stat_bin()` using `bins = 30`. Pick better value with `binwidth`.
## `stat_bin()` using `bins = 30`. Pick better value with `binwidth`.
## `stat_bin()` using `bins = 30`. Pick better value with `binwidth`.
\end{verbatim}

\includegraphics{STAT847_W24_Final_files/figure-latex/unnamed-chunk-5-1.pdf}

In question 2, We want to use ggpairs plot to analyze the shoting
performance of both teams (Team USA and Team Canada) in all four games.
So we will focus on either Goal: Successful Shot attempts (goal) and
Shot: Unsuccessful shot attempts (block, miss or save). As mentioned in
Q1, our goal is to track the number of shots attempted by each team to
show their ability to create scoring opportunities and the ability to
shot (finishing). In the meantime, it also can show the opponent's
goalkeeper's goalkeeping ability. We also will be able to identify the
high-danger areas where both teams are most effective at creating goal
opportunities.\\

Now let's talk about the six most important variables I will use to
generate such a ggpairs plot:\\

X Coordinate / Y Coordinate (Continuous): The X/Y coordinate provides
important information about their location (Event location) relative to
offensive, defensive, or neutral zones. In this case, we will focus on
the offensive. Analyzing this variable enhances our understanding of
team attacking/ shoting strategies and gameplay tactics. Also, we will
be able to know the area where they prefer to take a shot and may create
goal opportunities.\\

Period (Continuous): The period explains the current period when events
occur. It offers insights into game progression over time. Understanding
event distribution across periods helps assess team performance and
strategic adjustments throughout the game.\\

Event (Categorical): The event describes the type of event taking place
during the game, providing a detailed overview of the current key
moment. Analyzing event types facilitates the identification of
patterns, trends, and critical moments in the match. We will be focusing
on either goals or misses.\\

Team (Categorical): We need to focus on the two teams, Team USA and Team
Canada, this variable enables analysis of attacking/ shoting strategies
used by each team. By analyzing team-specific trends and behaviours, we
will be able to gain deeper insights into their performance and
competitive strategies.\\

Date (Categorical): There are 4 games in total in the series. Which is
the following:\\
2023-11-08, Home Team: USA, Away Team: Canada\\
2023-11-11, Home Team: USA, Away Team: Canada\\
2023-12-14, Home Team: Canada, Away Team: USA\\
2023-12-16, Home Team: Canada, Away Team: USA\\

By using the Date variables, we will be able to see the difference in
performance and attacking/ shoting strategies both teams use when they
are at a home game or an away game.\\

Let's take a look at the gg pairs plot in order to get a better
understanding of the performance of Team Canada and Team USA.\\

We differentiated each team by colour, with pink representing Canada and
blue representing the USA. Each row in the data set corresponds to
either a shot or a miss. They also indicate the number of attempts
(goals or misses) each team made.\\

Plot (1,1) shows the number of attempts per period. It's obvious that
Team Canada made more shoting attempts than Team USA across all four
periods. Especially leading in periods 2 and 4 (extra time).\\

Plot (2,2) shows the number of attempts per game (Date). We observe that
in away games (Canada's away game), Team Canada and Team USA created a
similar amount of opportunities. However, in home games, Team Canada
demonstrated a clear advantage in creating shoting opportunities.\\

Plot (1,2) and (2,1) explain attempts per period per game. Notably, Team
Canada had a higher number of shoting attempts in period 2. Team USA
only showed stronger offensive willingness in one or two periods in all
four games. Overall, the ggpairs suggest that Team Canada maintains a
more aggressive offensive strategy compared to Team USA.\\

A similar situation is shown in plot (5,5). It illustrates the number of
goals and shots created by both teams in all games. However, although
Team Canada had a higher number of attempts (shoting opportunities),
Team USA still scored more goals. This can indicate a potential weakness
in Team Canada's finishing abilities. The players waste a huge amount of
chances.\\

Further details are provided in plot (5,1). It shows the number of goals
and shots per period. Team Canada had a commendable performance in most
of period 2. It matches the goal rate of Team USA in period 2. However,
in periods 1, 3, and extra time, Team USA took the lead on the goals.\\

Plot (5,2) reconfirms Team Canada's advantage in creating shoting
opportunities. But it also highlights a concerning trend: although Team
Canada always has more shoting opportunities in nearly every game, Team
Canada failed to secure any victories, ending the series with 3 losses
and 1 draw. This suggests that Team USA possesses a stronger overall
performance in the series. It is also possible that Canada's strategy is
to play more offensively and shot whenever they can. But it doesn't have
any effective effect. More aggressive can cause the opponent to have a
chance to make a quick counterattack.~

Beyond goal-scoring statistics, we need to also analyze the spatial
distribution of attempts:\\

For Team USA:\\

X-axis analysis in column 3 indicates a preferred shoting area (global
maximum) just below 180 in plots (3,3) and (5,3). Similarly, the
majority of goals fall within the range of 170-190. Which is the area
pretty close to the goal gate.\\
Y-axis plots in column 4 have a peak around 45, with corresponding goals
clustered between 35-55 in plots (4,4) and (5,4). Which is in the middle
y-axis, which means facing the goal gate directly.\\

For Team Canada:\\

X-axis analysis in column 3 highlights multiple local maximums and a
global maximum at 180, with additional peaks at 140 and 160, indicating
preferred shoting positions. But most of the goals are still around
180.\\
The maximum of Y-axes in column 4 is also around 40 again, facing the
goal gate directly. But it has a much-flattened graph.\\
According to this result, we can also think maybe team Canada is trying
to score at a tricky angle, but barely succeeds.\\

In Q6, we will also having a ggplot which is about the offense
performance of both team, we will be taking about their game strategies
combine with ggplot graph and ggpairs.

\newpage

\vspace{2cm}

\begin{enumerate}
\def\labelenumi{\arabic{enumi})}
\setcounter{enumi}{2}
\tightlist
\item
  \textbf{MUST BE INCLUDED} Build a classification tree of one of the
  six variables from the last part as a function of the other five, and
  any other explanatory variables you think are necessary. Show code,
  explain reasoning, and show the tree as a simple (ugly) plot. Show the
  confusion matrix. Give two example predictions and follow them down
  the tree.
\end{enumerate}

\begin{Shaded}
\begin{Highlighting}[]
\CommentTok{\# split data}

\NormalTok{dat\_no\_pass }\OtherTok{\textless{}{-}} \FunctionTok{filter}\NormalTok{(dat, Event }\SpecialCharTok{!=} \StringTok{"Play"} \SpecialCharTok{\&}\NormalTok{ Event }\SpecialCharTok{!=} \StringTok{"Incomplete Play"}\NormalTok{)}
\NormalTok{dat\_split }\OtherTok{\textless{}{-}} \FunctionTok{initial\_split}\NormalTok{(dat\_no\_pass, }\AttributeTok{prop =} \FloatTok{0.8}\NormalTok{)}
\NormalTok{dat\_train }\OtherTok{\textless{}{-}} \FunctionTok{training}\NormalTok{(dat\_split)}
\NormalTok{dat\_test }\OtherTok{\textless{}{-}} \FunctionTok{testing}\NormalTok{(dat\_split) }

\NormalTok{fit }\OtherTok{=} \FunctionTok{rpart}\NormalTok{(Event}\SpecialCharTok{\textasciitilde{}}\NormalTok{Period }\SpecialCharTok{+}\NormalTok{ Date}\SpecialCharTok{+}\NormalTok{Y.Coordinate}\SpecialCharTok{+}\NormalTok{X.Coordinate }\SpecialCharTok{+}\NormalTok{ Team, }\AttributeTok{method=}\StringTok{"class"}\NormalTok{, }\AttributeTok{data=}\NormalTok{dat\_train)}


\CommentTok{\# Use pruning}
\NormalTok{pfit}\OtherTok{\textless{}{-}} \FunctionTok{prune}\NormalTok{(fit, }\AttributeTok{cp=}\NormalTok{ fit}\SpecialCharTok{$}\NormalTok{cptable[}\FunctionTok{which.min}\NormalTok{(fit}\SpecialCharTok{$}\NormalTok{cptable[,}\StringTok{"xerror"}\NormalTok{]),}\StringTok{"CP"}\NormalTok{])}
\FunctionTok{fancyRpartPlot}\NormalTok{(pfit, }\AttributeTok{main=}\StringTok{"Classification tree of Event"}\NormalTok{)}
\end{Highlighting}
\end{Shaded}

\includegraphics{STAT847_W24_Final_files/figure-latex/unnamed-chunk-6-1.pdf}

\begin{Shaded}
\begin{Highlighting}[]
\NormalTok{pred }\OtherTok{\textless{}{-}} \FunctionTok{predict}\NormalTok{(pfit, dat\_test, }\StringTok{"class"}\NormalTok{)}

\CommentTok{\# Make the confusion matrix}
\NormalTok{confusion }\OtherTok{\textless{}{-}} \FunctionTok{table}\NormalTok{(pred, dat\_test}\SpecialCharTok{$}\NormalTok{Event)}
\FunctionTok{print}\NormalTok{(confusion)}
\end{Highlighting}
\end{Shaded}

\begin{verbatim}
##                
## pred            Dump In/Out Faceoff Win Goal Penalty Taken Puck Recovery Shot
##   Dump In/Out             0           0    0             0             0    0
##   Faceoff Win             1          15    0             0             3    0
##   Goal                    0           0    0             0             0    0
##   Penalty Taken           0           0    0             0             0    0
##   Puck Recovery         113          12    0            11           416    8
##   Shot                    2          18    1             0            38   76
##   Takeaway                0           0    0             0             0    0
##   Zone Entry              0           0    0             0             2    0
##                
## pred            Takeaway Zone Entry
##   Dump In/Out          0          0
##   Faceoff Win          0          0
##   Goal                 0          0
##   Penalty Taken        0          0
##   Puck Recovery       43          1
##   Shot                 4          1
##   Takeaway             0          0
##   Zone Entry           0        106
\end{verbatim}

\begin{Shaded}
\begin{Highlighting}[]
\CommentTok{\# Print accuracy}
\NormalTok{accuracy }\OtherTok{\textless{}{-}} \FunctionTok{sum}\NormalTok{(}\FunctionTok{diag}\NormalTok{(confusion)) }\SpecialCharTok{/} \FunctionTok{sum}\NormalTok{(confusion)}
\FunctionTok{print}\NormalTok{(accuracy)}
\end{Highlighting}
\end{Shaded}

\begin{verbatim}
## [1] 0.7037887
\end{verbatim}

\begin{Shaded}
\begin{Highlighting}[]
\CommentTok{\# Exmaple predicions}

\CommentTok{\# 1}
\NormalTok{pred1 }\OtherTok{=} \FunctionTok{predict}\NormalTok{(pfit, dat\_test[}\DecValTok{1}\NormalTok{,], }\StringTok{"class"}\NormalTok{)}

\CommentTok{\# 2}
\NormalTok{pred2 }\OtherTok{=} \FunctionTok{predict}\NormalTok{(pfit, dat\_test[}\DecValTok{2}\NormalTok{,], }\StringTok{"class"}\NormalTok{)}
\FunctionTok{print}\NormalTok{(}\StringTok{"{-}{-}Example 1{-}{-}"}\NormalTok{)}
\end{Highlighting}
\end{Shaded}

\begin{verbatim}
## [1] "--Example 1--"
\end{verbatim}

\begin{Shaded}
\begin{Highlighting}[]
\FunctionTok{print}\NormalTok{(dat\_test[}\DecValTok{1}\NormalTok{, ])}
\end{Highlighting}
\end{Shaded}

\begin{verbatim}
##         Date             Home.Team      Away.Team Period Clock
## 1 2023-11-08 Women - United States Women - Canada      1 19:45
##   Home.Team.Skaters Away.Team.Skaters Home.Team.Goals Away.Team.Goals
## 1                 5                 5               0               0
##             Team      Player         Event X.Coordinate Y.Coordinate Detail.1
## 1 Women - Canada Renata Fast Puck Recovery            9            9         
##   Detail.2 Detail.3 Detail.4 Player.2 X.Coordinate.2 Y.Coordinate.2
## 1                                                 NA             NA
\end{verbatim}

\begin{Shaded}
\begin{Highlighting}[]
\FunctionTok{print}\NormalTok{(pred1)}
\end{Highlighting}
\end{Shaded}

\begin{verbatim}
##             1 
## Puck Recovery 
## 8 Levels: Dump In/Out Faceoff Win Goal Penalty Taken Puck Recovery ... Zone Entry
\end{verbatim}

\begin{Shaded}
\begin{Highlighting}[]
\FunctionTok{print}\NormalTok{(}\StringTok{"{-}{-}Example 2{-}{-}"}\NormalTok{)}
\end{Highlighting}
\end{Shaded}

\begin{verbatim}
## [1] "--Example 2--"
\end{verbatim}

\begin{Shaded}
\begin{Highlighting}[]
\FunctionTok{print}\NormalTok{(dat\_test[}\DecValTok{2}\NormalTok{, ])}
\end{Highlighting}
\end{Shaded}

\begin{verbatim}
##         Date             Home.Team      Away.Team Period Clock
## 2 2023-11-08 Women - United States Women - Canada      1 19:39
##   Home.Team.Skaters Away.Team.Skaters Home.Team.Goals Away.Team.Goals
## 2                 5                 5               0               0
##             Team              Player         Event X.Coordinate Y.Coordinate
## 2 Women - Canada Marie-Philip Poulin Puck Recovery           64           40
##   Detail.1 Detail.2 Detail.3 Detail.4 Player.2 X.Coordinate.2 Y.Coordinate.2
## 2                                                          NA             NA
\end{verbatim}

\begin{Shaded}
\begin{Highlighting}[]
\FunctionTok{print}\NormalTok{(pred2)}
\end{Highlighting}
\end{Shaded}

\begin{verbatim}
##             2 
## Puck Recovery 
## 8 Levels: Dump In/Out Faceoff Win Goal Penalty Taken Puck Recovery ... Zone Entry
\end{verbatim}

We aim to construct a classification tree to classify one of the six
variables from the previous question. We will choose the Event. Building
such a classification tree assists us in classifying the potential
attributes of each event.\\

We exclude passing actions from classification due to their high
frequency, location-free, and potential similarity with unsuccessful
passing. This exclusion prevents confusion in the classification tree
and ensures distinct attributes for each event category.\\

In further analysis, we can build a classification tree to classify the
passing location by teams, which allows us to find out if there's a
specific location where one of the teams likes to do the passing. But in
this question, We will focus solely on classifying non-passing events in
order to build a more general tree.\\

From the classification tree, it's evident that only the X-coordinate
and Y-coordinate are used. This could be due to a few reasons:\\

Let's consider other variables:\\

Date / Period: These variables represents the date and period of the
game, it's not a signiciant aid for classification since all events can
happens in all period/ games. The events listed all have high frequency
and they are most likely going to happens anytime during the game.\\

Team: Any team can trigger any event. Therefore, this variable does not
offer substantial value for classification purposes.\\

Hence, these variables are unlikely to significantly aid in the
construction of our classification tree.\\

\newpage

\vspace{2cm}

\begin{enumerate}
\def\labelenumi{\arabic{enumi})}
\setcounter{enumi}{3}
\tightlist
\item
  \textbf{MUST BE INCLUDED} Build another model using one of the
  continuous variables from your six most important. This time use your
  model selection and dimension reduction tools, and include at least
  one non-linear term.
\end{enumerate}

\begin{Shaded}
\begin{Highlighting}[]
\CommentTok{\# prepare data}
\NormalTok{model\_dat }\OtherTok{\textless{}{-}}\NormalTok{ dat }\SpecialCharTok{\%\textgreater{}\%} \FunctionTok{group\_by}\NormalTok{(Team,Period, X.Coordinate, Event) }\SpecialCharTok{\%\textgreater{}\%} \FunctionTok{summarise}\NormalTok{(}\AttributeTok{num\_event =} \FunctionTok{n}\NormalTok{())}
\end{Highlighting}
\end{Shaded}

\begin{verbatim}
## `summarise()` has grouped output by 'Team', 'Period', 'X.Coordinate'. You can
## override using the `.groups` argument.
\end{verbatim}

\begin{Shaded}
\begin{Highlighting}[]
\NormalTok{dat\_split }\OtherTok{\textless{}{-}} \FunctionTok{initial\_split}\NormalTok{(model\_dat, }\AttributeTok{prop =} \FloatTok{0.8}\NormalTok{)}
\NormalTok{dat\_train }\OtherTok{\textless{}{-}} \FunctionTok{training}\NormalTok{(dat\_split)}
\NormalTok{dat\_test }\OtherTok{\textless{}{-}} \FunctionTok{testing}\NormalTok{(dat\_split) }
\end{Highlighting}
\end{Shaded}

\begin{Shaded}
\begin{Highlighting}[]
\NormalTok{lm\_model }\OtherTok{\textless{}{-}} \FunctionTok{lm}\NormalTok{(dat\_train}\SpecialCharTok{$}\NormalTok{num\_event}\SpecialCharTok{\textasciitilde{}} \FunctionTok{I}\NormalTok{(X.Coordinate}\SpecialCharTok{\^{}}\DecValTok{2}\NormalTok{) }\SpecialCharTok{+}\NormalTok{., }\AttributeTok{data =}\NormalTok{dat\_train)}

\CommentTok{\# Use Stepwise regression}
\NormalTok{step }\OtherTok{\textless{}{-}} \FunctionTok{step}\NormalTok{(lm\_model, }\AttributeTok{direction=}\StringTok{"both"}\NormalTok{)}
\end{Highlighting}
\end{Shaded}

\begin{verbatim}
## Start:  AIC=4505.98
## dat_train$num_event ~ I(X.Coordinate^2) + (Team + Period + X.Coordinate + 
##     Event)
## 
##                     Df Sum of Sq   RSS    AIC
## - Team               1       0.2 13896 4504.0
## <none>                           13896 4506.0
## - Period             3     330.1 14226 4565.3
## - I(X.Coordinate^2)  1     607.9 14504 4623.0
## - X.Coordinate       1     773.9 14670 4654.7
## - Event              9    8803.4 22699 5852.7
## 
## Step:  AIC=4504.02
## dat_train$num_event ~ I(X.Coordinate^2) + Period + X.Coordinate + 
##     Event
## 
##                     Df Sum of Sq   RSS    AIC
## <none>                           13896 4504.0
## + Team               1       0.2 13896 4506.0
## - Period             3     330.0 14226 4563.3
## - I(X.Coordinate^2)  1     608.0 14504 4621.1
## - X.Coordinate       1     774.2 14670 4652.8
## - Event              9    8803.3 22699 5850.8
\end{verbatim}

\begin{Shaded}
\begin{Highlighting}[]
\FunctionTok{summary}\NormalTok{(step)}
\end{Highlighting}
\end{Shaded}

\begin{verbatim}
## 
## Call:
## lm(formula = dat_train$num_event ~ I(X.Coordinate^2) + Period + 
##     X.Coordinate + Event, data = dat_train)
## 
## Residuals:
##      Min       1Q   Median       3Q      Max 
## -18.4927  -0.9233  -0.1913   0.5373  31.5250 
## 
## Coefficients:
##                        Estimate Std. Error t value Pr(>|t|)    
## (Intercept)           3.039e+00  1.826e-01  16.646  < 2e-16 ***
## I(X.Coordinate^2)     1.581e-04  1.437e-05  11.001  < 2e-16 ***
## Period2               2.313e-02  1.050e-01   0.220   0.8256    
## Period3              -1.003e-02  1.060e-01  -0.095   0.9246    
## Period4              -2.048e+00  2.604e-01  -7.865 5.25e-15 ***
## X.Coordinate         -3.601e-02  2.901e-03 -12.414  < 2e-16 ***
## EventFaceoff Win      6.219e+00  4.578e-01  13.584  < 2e-16 ***
## EventGoal            -3.974e-01  5.819e-01  -0.683   0.4947    
## EventIncomplete Play -2.182e-01  1.693e-01  -1.289   0.1976    
## EventPenalty Taken   -7.318e-01  4.436e-01  -1.650   0.0991 .  
## EventPlay             8.956e-01  1.513e-01   5.918 3.65e-09 ***
## EventPuck Recovery    8.187e-01  1.526e-01   5.366 8.73e-08 ***
## EventShot             2.494e-01  2.149e-01   1.161   0.2458    
## EventTakeaway        -5.294e-01  2.112e-01  -2.507   0.0122 *  
## EventZone Entry       1.848e+01  4.881e-01  37.850  < 2e-16 ***
## ---
## Signif. codes:  0 '***' 0.001 '**' 0.01 '*' 0.05 '.' 0.1 ' ' 1
## 
## Residual standard error: 2.241 on 2766 degrees of freedom
## Multiple R-squared:  0.4003, Adjusted R-squared:  0.3973 
## F-statistic: 131.9 on 14 and 2766 DF,  p-value: < 2.2e-16
\end{verbatim}

\begin{Shaded}
\begin{Highlighting}[]
\CommentTok{\# mean squared error}
\NormalTok{test }\OtherTok{\textless{}{-}} \FunctionTok{data.frame}\NormalTok{(}\FunctionTok{predict}\NormalTok{(step, dat\_test), }\AttributeTok{actual =}\NormalTok{ dat\_test}\SpecialCharTok{$}\NormalTok{num\_event)}
\FunctionTok{mean}\NormalTok{((test}\SpecialCharTok{$}\NormalTok{actual }\SpecialCharTok{{-}}\NormalTok{ test}\SpecialCharTok{$}\NormalTok{pred)}\SpecialCharTok{\^{}}\DecValTok{2}\NormalTok{)}
\end{Highlighting}
\end{Shaded}

\begin{verbatim}
## [1] 3.035467
\end{verbatim}

\begin{Shaded}
\begin{Highlighting}[]
\CommentTok{\# Use best subsets regression with the Adjusted R{-}squared criterion}

\NormalTok{regsubsetsObj }\OtherTok{\textless{}{-}} \FunctionTok{regsubsets}\NormalTok{(dat\_train}\SpecialCharTok{$}\NormalTok{num\_event}\SpecialCharTok{\textasciitilde{}} \FunctionTok{I}\NormalTok{(X.Coordinate}\SpecialCharTok{\^{}}\DecValTok{2}\NormalTok{) }\SpecialCharTok{+}\NormalTok{ ., }\AttributeTok{data=}\NormalTok{dat\_train, }\AttributeTok{really.big=}\ConstantTok{TRUE}\NormalTok{)}
\FunctionTok{print}\NormalTok{(}\FunctionTok{summary}\NormalTok{(regsubsetsObj)}\SpecialCharTok{$}\NormalTok{adjr2)}
\end{Highlighting}
\end{Shaded}

\begin{verbatim}
## [1] 0.2874915 0.3215921 0.3324492 0.3559632 0.3670528 0.3815303 0.3955162
## [8] 0.3968622
\end{verbatim}

\begin{Shaded}
\begin{Highlighting}[]
\NormalTok{coef\_list }\OtherTok{=} \FunctionTok{coef}\NormalTok{(regsubsetsObj,}\DecValTok{8}\NormalTok{)}
\FunctionTok{print}\NormalTok{(coef\_list)}
\end{Highlighting}
\end{Shaded}

\begin{verbatim}
##        (Intercept)  I(X.Coordinate^2)            Period4       X.Coordinate 
##       2.7999055877       0.0001523115      -2.0655980621      -0.0350225213 
##   EventFaceoff Win          EventPlay EventPuck Recovery          EventShot 
##       6.4366905185       1.1198904027       1.0437527167       0.4875967598 
##    EventZone Entry 
##      18.6871943671
\end{verbatim}

\begin{Shaded}
\begin{Highlighting}[]
\NormalTok{regsubsets\_formula }\OtherTok{=} \StringTok{"dat\_train$num\_event\textasciitilde{}"}

\NormalTok{coef\_list }\OtherTok{=}\NormalTok{ coef\_list[}\SpecialCharTok{{-}}\DecValTok{1}\NormalTok{]}
\ControlFlowTok{for}\NormalTok{ (name }\ControlFlowTok{in} \FunctionTok{names}\NormalTok{(coef\_list)) \{}
\NormalTok{  value }\OtherTok{\textless{}{-}}\NormalTok{ coef\_list[[name]]}
  \ControlFlowTok{if}\NormalTok{ (value }\SpecialCharTok{!=} \DecValTok{0}\NormalTok{)\{}
\NormalTok{    name }\OtherTok{\textless{}{-}} \FunctionTok{ifelse}\NormalTok{(}\FunctionTok{grepl}\NormalTok{(}\StringTok{"\^{}Period"}\NormalTok{, name), }\StringTok{"Period"}\NormalTok{, name)}
\NormalTok{    name }\OtherTok{\textless{}{-}} \FunctionTok{ifelse}\NormalTok{(}\FunctionTok{grepl}\NormalTok{(}\StringTok{"\^{}Event"}\NormalTok{, name), }\StringTok{"Event"}\NormalTok{, name)}
\NormalTok{    regsubsets\_formula }\OtherTok{=} \FunctionTok{paste}\NormalTok{(regsubsets\_formula, name)}
\NormalTok{    regsubsets\_formula }\OtherTok{=} \FunctionTok{paste}\NormalTok{(regsubsets\_formula, }\StringTok{"+"}\NormalTok{)}
\NormalTok{  \}}
\NormalTok{\}}
\NormalTok{regsubsets\_formula }\OtherTok{\textless{}{-}} \FunctionTok{substring}\NormalTok{(regsubsets\_formula, }\DecValTok{1}\NormalTok{, }\FunctionTok{nchar}\NormalTok{(regsubsets\_formula) }\SpecialCharTok{{-}} \DecValTok{1}\NormalTok{)}

\CommentTok{\# print(regsubsets\_formula)}
\NormalTok{best\_sub\_model }\OtherTok{=} \FunctionTok{lm}\NormalTok{(}\FunctionTok{as.formula}\NormalTok{(regsubsets\_formula), }\AttributeTok{data =}\NormalTok{ dat\_train)}

\FunctionTok{summary}\NormalTok{(best\_sub\_model)}
\end{Highlighting}
\end{Shaded}

\begin{verbatim}
## 
## Call:
## lm(formula = as.formula(regsubsets_formula), data = dat_train)
## 
## Residuals:
##      Min       1Q   Median       3Q      Max 
## -18.4927  -0.9233  -0.1913   0.5373  31.5250 
## 
## Coefficients:
##                        Estimate Std. Error t value Pr(>|t|)    
## (Intercept)           3.039e+00  1.826e-01  16.646  < 2e-16 ***
## I(X.Coordinate^2)     1.581e-04  1.437e-05  11.001  < 2e-16 ***
## Period2               2.313e-02  1.050e-01   0.220   0.8256    
## Period3              -1.003e-02  1.060e-01  -0.095   0.9246    
## Period4              -2.048e+00  2.604e-01  -7.865 5.25e-15 ***
## X.Coordinate         -3.601e-02  2.901e-03 -12.414  < 2e-16 ***
## EventFaceoff Win      6.219e+00  4.578e-01  13.584  < 2e-16 ***
## EventGoal            -3.974e-01  5.819e-01  -0.683   0.4947    
## EventIncomplete Play -2.182e-01  1.693e-01  -1.289   0.1976    
## EventPenalty Taken   -7.318e-01  4.436e-01  -1.650   0.0991 .  
## EventPlay             8.956e-01  1.513e-01   5.918 3.65e-09 ***
## EventPuck Recovery    8.187e-01  1.526e-01   5.366 8.73e-08 ***
## EventShot             2.494e-01  2.149e-01   1.161   0.2458    
## EventTakeaway        -5.294e-01  2.112e-01  -2.507   0.0122 *  
## EventZone Entry       1.848e+01  4.881e-01  37.850  < 2e-16 ***
## ---
## Signif. codes:  0 '***' 0.001 '**' 0.01 '*' 0.05 '.' 0.1 ' ' 1
## 
## Residual standard error: 2.241 on 2766 degrees of freedom
## Multiple R-squared:  0.4003, Adjusted R-squared:  0.3973 
## F-statistic: 131.9 on 14 and 2766 DF,  p-value: < 2.2e-16
\end{verbatim}

\begin{Shaded}
\begin{Highlighting}[]
\CommentTok{\# mean squared error}
\NormalTok{test }\OtherTok{\textless{}{-}} \FunctionTok{data.frame}\NormalTok{(}\FunctionTok{predict}\NormalTok{(best\_sub\_model, dat\_test), }\AttributeTok{actual =}\NormalTok{ dat\_test}\SpecialCharTok{$}\NormalTok{num\_event)}
\FunctionTok{mean}\NormalTok{((test}\SpecialCharTok{$}\NormalTok{actual }\SpecialCharTok{{-}}\NormalTok{ test}\SpecialCharTok{$}\NormalTok{pred)}\SpecialCharTok{\^{}}\DecValTok{2}\NormalTok{)}
\end{Highlighting}
\end{Shaded}

\begin{verbatim}
## [1] 3.035467
\end{verbatim}

\begin{Shaded}
\begin{Highlighting}[]
\CommentTok{\# Use Dimension reduction tool}
\NormalTok{famd }\OtherTok{\textless{}{-}} \FunctionTok{FAMD}\NormalTok{(dat\_train,}\AttributeTok{ncp=}\DecValTok{3}\NormalTok{, }\AttributeTok{graph=}\ConstantTok{FALSE}\NormalTok{)}

\CommentTok{\# Extract the components}
\NormalTok{reduced }\OtherTok{\textless{}{-}} \FunctionTok{as.data.frame}\NormalTok{(famd}\SpecialCharTok{$}\NormalTok{ind}\SpecialCharTok{$}\NormalTok{coord)}

\CommentTok{\# Build a linear model }
\NormalTok{famd\_model }\OtherTok{\textless{}{-}} \FunctionTok{lm}\NormalTok{(dat\_train}\SpecialCharTok{$}\NormalTok{num\_event}\SpecialCharTok{\textasciitilde{}} \FunctionTok{I}\NormalTok{(Dim}\FloatTok{.1}\SpecialCharTok{\^{}}\DecValTok{2}\NormalTok{) }\SpecialCharTok{+}\NormalTok{ ., }\AttributeTok{data =}\NormalTok{ reduced )}

\CommentTok{\# Print the summary of the linear model}
\FunctionTok{summary}\NormalTok{(famd\_model)}
\end{Highlighting}
\end{Shaded}

\begin{verbatim}
## 
## Call:
## lm(formula = dat_train$num_event ~ I(Dim.1^2) + ., data = reduced)
## 
## Residuals:
##      Min       1Q   Median       3Q      Max 
## -14.0736  -0.4746  -0.1164   0.3304  10.0339 
## 
## Coefficients:
##              Estimate Std. Error t value Pr(>|t|)    
## (Intercept)  2.137579   0.024547  87.082  < 2e-16 ***
## I(Dim.1^2)   0.023061   0.003109   7.417 1.58e-13 ***
## Dim.1        1.782341   0.038862  45.863  < 2e-16 ***
## Dim.2        0.011092   0.021460   0.517    0.605    
## Dim.3       -0.250660   0.023583 -10.629  < 2e-16 ***
## ---
## Signif. codes:  0 '***' 0.001 '**' 0.01 '*' 0.05 '.' 0.1 ' ' 1
## 
## Residual standard error: 1.268 on 2776 degrees of freedom
## Multiple R-squared:  0.8075, Adjusted R-squared:  0.8072 
## F-statistic:  2911 on 4 and 2776 DF,  p-value: < 2.2e-16
\end{verbatim}

\begin{Shaded}
\begin{Highlighting}[]
\CommentTok{\# Use Dimension reduction tool}
\NormalTok{famd\_test }\OtherTok{\textless{}{-}} \FunctionTok{FAMD}\NormalTok{(dat\_test,}\AttributeTok{ncp=}\DecValTok{3}\NormalTok{, }\AttributeTok{graph=}\ConstantTok{FALSE}\NormalTok{)}

\CommentTok{\# Extract the components}
\NormalTok{reduced\_test }\OtherTok{\textless{}{-}} \FunctionTok{as.data.frame}\NormalTok{(famd\_test}\SpecialCharTok{$}\NormalTok{ind}\SpecialCharTok{$}\NormalTok{coord)}

\CommentTok{\# mean squared error}
\NormalTok{test }\OtherTok{\textless{}{-}} \FunctionTok{data.frame}\NormalTok{(}\FunctionTok{predict}\NormalTok{(famd\_model, reduced\_test), }\AttributeTok{actual =}\NormalTok{ dat\_test}\SpecialCharTok{$}\NormalTok{num\_event)}
\FunctionTok{mean}\NormalTok{((test}\SpecialCharTok{$}\NormalTok{actual }\SpecialCharTok{{-}}\NormalTok{ test}\SpecialCharTok{$}\NormalTok{pred)}\SpecialCharTok{\^{}}\DecValTok{2}\NormalTok{)}
\end{Highlighting}
\end{Shaded}

\begin{verbatim}
## [1] 1.586788
\end{verbatim}

In this question, we are constructing a model using one of the
continuous variables among my six most important ones, which is
X-Coordinate. The model predicts the number of events at a given Period,
Team, and X-coordinate. We build three models using stepwise regression,
best-subset, and Factor Analysis of Mixed Data. By calculating mean
square error on the test set, we find that the RAMD model has the
smallest error, and it's simpler with reduced dimensionality to 3. Using
this model, we can predict and learn about each team's strategy. As the
model learns each team's habits, we'll know what events are likely to
happen across all x-axes for each team and the estimated number of
events in specific locations during specific periods. This is valuable
for estimating each team's performance and strategies.

\newpage

\vspace{2cm}

\begin{enumerate}
\def\labelenumi{\arabic{enumi})}
\setcounter{enumi}{5}
\tightlist
\item
  \textbf{OPTIONAL: PICK 2 OF 4} Build a visually impressive ggplot to
  show the relationship between at least three variables.
\end{enumerate}

\begin{Shaded}
\begin{Highlighting}[]
\NormalTok{dat6 }\OtherTok{=}\NormalTok{ dat}

\NormalTok{time\_parts }\OtherTok{\textless{}{-}} \FunctionTok{strsplit}\NormalTok{(dat6}\SpecialCharTok{$}\NormalTok{Clock, }\StringTok{":"}\NormalTok{)}





\NormalTok{clock\_to\_sec }\OtherTok{\textless{}{-}} \ControlFlowTok{function}\NormalTok{(clock) \{}
\NormalTok{  clock\_parts }\OtherTok{\textless{}{-}} \FunctionTok{strsplit}\NormalTok{(clock, }\StringTok{":"}\NormalTok{)}
\NormalTok{  minutes }\OtherTok{\textless{}{-}} \FunctionTok{as.numeric}\NormalTok{(clock\_parts[[}\DecValTok{1}\NormalTok{]][}\DecValTok{1}\NormalTok{])}
\NormalTok{  seconds }\OtherTok{\textless{}{-}} \FunctionTok{as.numeric}\NormalTok{(clock\_parts[[}\DecValTok{1}\NormalTok{]][}\DecValTok{2}\NormalTok{])}
  \FunctionTok{return}\NormalTok{(}\DecValTok{1200} \SpecialCharTok{{-}}\NormalTok{ (minutes }\SpecialCharTok{*} \DecValTok{60} \SpecialCharTok{+}\NormalTok{ seconds))}
\NormalTok{\}}

\CommentTok{\# Change the clock to second and start from 00:00 instead of 20:00}
\NormalTok{dat6}\SpecialCharTok{$}\NormalTok{Clock\_sec }\OtherTok{\textless{}{-}} \FunctionTok{as.numeric}\NormalTok{(}\FunctionTok{sapply}\NormalTok{(dat6}\SpecialCharTok{$}\NormalTok{Clock, clock\_to\_sec) }\SpecialCharTok{+}\NormalTok{ (}\FunctionTok{as.numeric}\NormalTok{(dat6}\SpecialCharTok{$}\NormalTok{Period) }\SpecialCharTok{{-}} \DecValTok{1}\NormalTok{) }\SpecialCharTok{*} \DecValTok{1200}\NormalTok{ )}


\CommentTok{\# Make a column tracking the shoting event}
\NormalTok{make\_cont }\OtherTok{\textless{}{-}} \ControlFlowTok{function}\NormalTok{(shot\_count, team, game\_date) \{}
\NormalTok{  curr }\OtherTok{\textless{}{-}} \DecValTok{0}
\NormalTok{  curr\_Date }\OtherTok{=} \StringTok{"2023{-}11{-}08"}
  \ControlFlowTok{for}\NormalTok{ (i }\ControlFlowTok{in} \DecValTok{1}\SpecialCharTok{:}\FunctionTok{length}\NormalTok{(shot\_count)) \{}
    \ControlFlowTok{if}\NormalTok{ (shot\_count[i] }\SpecialCharTok{!=} \SpecialCharTok{{-}}\DecValTok{1}\NormalTok{) \{}
\NormalTok{      curr }\OtherTok{\textless{}{-}}\NormalTok{ shot\_count[i]}
      \ControlFlowTok{if}\NormalTok{ (game\_date[i] }\SpecialCharTok{==} \StringTok{"2023{-}11{-}11"}\NormalTok{)\{}
        \ControlFlowTok{if}\NormalTok{ (team }\SpecialCharTok{==} \StringTok{"Women {-} Canada"}\NormalTok{)\{}
\NormalTok{          curr }\OtherTok{\textless{}{-}}\NormalTok{ curr }\SpecialCharTok{{-}} \DecValTok{55}
\NormalTok{        \}}\ControlFlowTok{else}\NormalTok{\{curr }\OtherTok{\textless{}{-}}\NormalTok{ curr }\SpecialCharTok{{-}} \DecValTok{56}\NormalTok{\}}
\NormalTok{      \}}
      \ControlFlowTok{if}\NormalTok{ (game\_date[i] }\SpecialCharTok{==} \StringTok{"2023{-}12{-}14"}\NormalTok{)\{}
        \ControlFlowTok{if}\NormalTok{ (team }\SpecialCharTok{==} \StringTok{"Women {-} Canada"}\NormalTok{)\{}
\NormalTok{          curr }\OtherTok{\textless{}{-}}\NormalTok{ curr }\SpecialCharTok{{-}} \DecValTok{105}
\NormalTok{        \}}\ControlFlowTok{else}\NormalTok{\{curr }\OtherTok{\textless{}{-}}\NormalTok{ curr }\SpecialCharTok{{-}} \DecValTok{95}\NormalTok{\}}
\NormalTok{      \}}
      \ControlFlowTok{if}\NormalTok{ (game\_date[i] }\SpecialCharTok{==} \StringTok{"2023{-}12{-}16"}\NormalTok{)\{}
        \ControlFlowTok{if}\NormalTok{ (team }\SpecialCharTok{==} \StringTok{"Women {-} Canada"}\NormalTok{)\{}
\NormalTok{          curr }\OtherTok{\textless{}{-}}\NormalTok{ curr }\SpecialCharTok{{-}} \DecValTok{162}
\NormalTok{        \}}\ControlFlowTok{else}\NormalTok{\{curr }\OtherTok{\textless{}{-}}\NormalTok{ curr }\SpecialCharTok{{-}} \DecValTok{138}\NormalTok{\}}
\NormalTok{      \}}
\NormalTok{      shot\_count[i] }\OtherTok{\textless{}{-}}\NormalTok{ curr}
\NormalTok{      curr\_Date }\OtherTok{=}\NormalTok{ game\_date[i]}
\NormalTok{    \} }\ControlFlowTok{else}\NormalTok{ \{}
      \ControlFlowTok{if}\NormalTok{ (game\_date[i] }\SpecialCharTok{!=}\NormalTok{ curr\_Date)\{}
\NormalTok{        shot\_count[i] }\OtherTok{\textless{}{-}} \DecValTok{0}\NormalTok{\}}
      \ControlFlowTok{else}\NormalTok{\{shot\_count[i] }\OtherTok{\textless{}{-}}\NormalTok{ curr\}}
      
\NormalTok{    \}}
\NormalTok{  \}}
  
  \FunctionTok{return}\NormalTok{(shot\_count)}
\NormalTok{\}}


\CommentTok{\# cumsum the shots for both team}
\NormalTok{dat6 }\OtherTok{\textless{}{-}}\NormalTok{ dat6 }\SpecialCharTok{\%\textgreater{}\%}
  \FunctionTok{arrange}\NormalTok{(Date, Clock\_sec) }\SpecialCharTok{\%\textgreater{}\%} 
  \FunctionTok{mutate}\NormalTok{(}\AttributeTok{Shoot\_Count\_ca =} \FunctionTok{ifelse}\NormalTok{(Team }\SpecialCharTok{==} \StringTok{"Women {-} Canada"} \SpecialCharTok{\&}\NormalTok{ Event }\SpecialCharTok{==} \StringTok{"Shot"}\NormalTok{, }\FunctionTok{cumsum}\NormalTok{(Event }\SpecialCharTok{==} \StringTok{"Shot"} \SpecialCharTok{\&}\NormalTok{ Team }\SpecialCharTok{==} \StringTok{"Women {-} Canada"}\NormalTok{), }\SpecialCharTok{{-}}\DecValTok{1}\NormalTok{))}


\NormalTok{dat6 }\OtherTok{\textless{}{-}}\NormalTok{ dat6 }\SpecialCharTok{\%\textgreater{}\%}
  \FunctionTok{arrange}\NormalTok{(Date, Clock\_sec) }\SpecialCharTok{\%\textgreater{}\%} 
  \FunctionTok{mutate}\NormalTok{(}\AttributeTok{Shoot\_Count\_us =} \FunctionTok{ifelse}\NormalTok{(Team }\SpecialCharTok{==} \StringTok{"Women {-} United States"} \SpecialCharTok{\&}\NormalTok{ Event }\SpecialCharTok{==} \StringTok{"Shot"}\NormalTok{, }\FunctionTok{cumsum}\NormalTok{(Event }\SpecialCharTok{==} \StringTok{"Shot"} \SpecialCharTok{\&}\NormalTok{ Team }\SpecialCharTok{==} \StringTok{"Women {-} United States"}\NormalTok{), }\SpecialCharTok{{-}}\DecValTok{1}\NormalTok{))}

\NormalTok{dat6}\SpecialCharTok{$}\NormalTok{Shoot\_Count\_ca }\OtherTok{=} \FunctionTok{make\_cont}\NormalTok{(dat6}\SpecialCharTok{$}\NormalTok{Shoot\_Count\_ca, }\StringTok{"Women {-} Canada"}\NormalTok{,dat6}\SpecialCharTok{$}\NormalTok{Date)}
\NormalTok{dat6}\SpecialCharTok{$}\NormalTok{Shoot\_Count\_us }\OtherTok{=} \FunctionTok{make\_cont}\NormalTok{(dat6}\SpecialCharTok{$}\NormalTok{Shoot\_Count\_us,}\StringTok{"Women {-} United States"}\NormalTok{,dat6}\SpecialCharTok{$}\NormalTok{Date)}

\FunctionTok{ggplot}\NormalTok{(dat6, }\FunctionTok{aes}\NormalTok{(}\AttributeTok{x =}\NormalTok{ Clock\_sec)) }\SpecialCharTok{+} 
  \FunctionTok{geom\_line}\NormalTok{(}\FunctionTok{aes}\NormalTok{(}\AttributeTok{y =}\NormalTok{ Shoot\_Count\_ca, }\AttributeTok{color =} \StringTok{"Canada\_shots"}\NormalTok{), }\AttributeTok{size =} \FloatTok{1.5}\NormalTok{) }\SpecialCharTok{+}
  \FunctionTok{geom\_line}\NormalTok{(}\FunctionTok{aes}\NormalTok{(}\AttributeTok{y =}\NormalTok{ Shoot\_Count\_us, }\AttributeTok{color =} \StringTok{"USA\_shots"}\NormalTok{), }\AttributeTok{size =} \FloatTok{1.5}\NormalTok{) }\SpecialCharTok{+}
  
   \FunctionTok{geom\_line}\NormalTok{(}\FunctionTok{aes}\NormalTok{(}\AttributeTok{y =}\NormalTok{ Home.Team.Goals }\SpecialCharTok{*} \DecValTok{9}\NormalTok{, }\AttributeTok{color =} \FunctionTok{ifelse}\NormalTok{(Home.Team }\SpecialCharTok{==} \StringTok{"Women {-} Canada"}\NormalTok{, }\StringTok{"Canada\_Home"}\NormalTok{, }\StringTok{"USA\_Home"}\NormalTok{)), }\AttributeTok{size =} \FloatTok{1.5}\NormalTok{) }\SpecialCharTok{+}
  
  \FunctionTok{geom\_line}\NormalTok{(}\FunctionTok{aes}\NormalTok{(}\AttributeTok{y =}\NormalTok{ Away.Team.Goals }\SpecialCharTok{*} \DecValTok{9}\NormalTok{, }\AttributeTok{color =} \FunctionTok{ifelse}\NormalTok{(Away.Team }\SpecialCharTok{==} \StringTok{"Women {-} Canada"}\NormalTok{, }\StringTok{"Canada\_Away"}\NormalTok{, }\StringTok{"USA\_Away"}\NormalTok{)), }\AttributeTok{size =} \FloatTok{1.5}\NormalTok{) }\SpecialCharTok{+}
  
    \FunctionTok{scale\_y\_continuous}\NormalTok{(}
    
    \AttributeTok{name =} \StringTok{"Number of Shots"}\NormalTok{,}
    
    \AttributeTok{sec.axis =} \FunctionTok{sec\_axis}\NormalTok{(}\SpecialCharTok{\textasciitilde{}}\NormalTok{.}\SpecialCharTok{/}\DecValTok{9}\NormalTok{, }\AttributeTok{name=}\StringTok{"Number of Goals"}\NormalTok{)}
\NormalTok{  ) }\SpecialCharTok{+} 
  
  
  \FunctionTok{facet\_wrap}\NormalTok{(}\SpecialCharTok{\textasciitilde{}}\NormalTok{ Date, }\AttributeTok{nrow =} \DecValTok{2}\NormalTok{, }\AttributeTok{scales =} \StringTok{\textquotesingle{}free\_y\textquotesingle{}}\NormalTok{) }\SpecialCharTok{+}
  \FunctionTok{xlim}\NormalTok{(}\DecValTok{0}\NormalTok{,}\DecValTok{4800}\NormalTok{)}\SpecialCharTok{+}
  \FunctionTok{scale\_color\_manual}\NormalTok{(}\AttributeTok{values =} \FunctionTok{c}\NormalTok{(}\AttributeTok{Canada\_Home =} \StringTok{"red"}\NormalTok{, }\AttributeTok{Canada\_Away =} \StringTok{"lightcoral"}\NormalTok{,}
                               \AttributeTok{USA\_Home =} \StringTok{"blue"}\NormalTok{, }\AttributeTok{USA\_Away =} \StringTok{"lightblue4"}\NormalTok{, }
                               \AttributeTok{Canada\_shots =} \StringTok{"pink1"}\NormalTok{, }\AttributeTok{USA\_shots =} \StringTok{"lightskyblue2"}\NormalTok{,}
                               \AttributeTok{Other =} \StringTok{"black"}\NormalTok{)) }\SpecialCharTok{+}
\FunctionTok{labs}\NormalTok{(}\AttributeTok{title =} \StringTok{"Goals Comparison between Teams"}\NormalTok{, }\AttributeTok{x =} \StringTok{"Clock Time"}\NormalTok{, }\AttributeTok{y =} \StringTok{"Number of Goals"}\NormalTok{, }\AttributeTok{color =} \StringTok{"Team"}\NormalTok{) }\SpecialCharTok{+} 
  \FunctionTok{geom\_vline}\NormalTok{(}\AttributeTok{xintercept =} \FunctionTok{c}\NormalTok{(}\DecValTok{0}\NormalTok{, }\DecValTok{1200}\NormalTok{, }\DecValTok{2400}\NormalTok{, }\DecValTok{3600}\NormalTok{, }\DecValTok{4800}\NormalTok{), }\AttributeTok{linetype =} \StringTok{"dashed"}\NormalTok{, }\AttributeTok{color =} \StringTok{"black"}\NormalTok{)}\SpecialCharTok{+}
  \FunctionTok{theme\_minimal}\NormalTok{() }\SpecialCharTok{+}
  \FunctionTok{theme}\NormalTok{(}
    \AttributeTok{plot.title =} \FunctionTok{element\_text}\NormalTok{(}\AttributeTok{size =} \DecValTok{20}\NormalTok{, }\AttributeTok{face =} \StringTok{"bold"}\NormalTok{),}
    \AttributeTok{axis.title =} \FunctionTok{element\_text}\NormalTok{(}\AttributeTok{size =} \DecValTok{14}\NormalTok{),}
    \AttributeTok{legend.title =} \FunctionTok{element\_text}\NormalTok{(}\AttributeTok{size =} \DecValTok{12}\NormalTok{),}
    \AttributeTok{legend.text =} \FunctionTok{element\_text}\NormalTok{(}\AttributeTok{size =} \DecValTok{10}\NormalTok{),}
    \AttributeTok{panel.grid.minor =} \FunctionTok{element\_blank}\NormalTok{(),}
    \AttributeTok{panel.grid.major =} \FunctionTok{element\_line}\NormalTok{(}\AttributeTok{color =} \StringTok{"gray"}\NormalTok{, }\AttributeTok{linetype =} \StringTok{"dashed"}\NormalTok{)}
\NormalTok{  )}
\end{Highlighting}
\end{Shaded}

\begin{verbatim}
## Warning: Using `size` aesthetic for lines was deprecated in ggplot2 3.4.0.
## i Please use `linewidth` instead.
## This warning is displayed once every 8 hours.
## Call `lifecycle::last_lifecycle_warnings()` to see where this warning was
## generated.
\end{verbatim}

\includegraphics{STAT847_W24_Final_files/figure-latex/unnamed-chunk-11-1.pdf}

The plot illustrates the correlation among Clock Time, Period, Date,
Home Team, Away Team, Home Team Goals, and Away Team Goals.Also with the
number of shots both team have.\\

Clock Time is derived from the clock and period columns, where the clock
counts down while the clock time counts up. Each period spans 20
minutes, equivalent to 1200 seconds. Thus, the current clock time is
calculated as 1200 multiplied by the period number minus one, plus the
clock time counting up. The dotted line denotes the period divisions.
The labels distinguish between the Team USA and Team Canada, either it's
home or away, and the amount of attempts both teams do in order to get
goal. Left y-axis stands for number of shots each team made, right
y-axis stands for number of goals each team have by making these amount
of shots.\\

From the graph, we obtained similar results as ggpairs, revealing that
although Team Canada has a significant advantage in shot quantity, it
doesn't translate into goals. In the Dec.16 game, despite Canada's
efforts to score in the second and third periods, numerous attempts only
led to Team USA extending their lead.\\

We can also analyze both teams' attacking strategies using this plot.\\

For Team USA:\\
In their Home Game, the shot count rises steadily when the score is
close. Once they lead by a lot, they become more aggressive.
Particularly in game 2, with a substantial lead, attempts increase
notably in period 3.\\
In their Away Game, shots increase steadily but decrease toward the end.
This can be because they are prioritizing defence over offence,
regardless of the score. They prefer not to lose than win the game.\\

Team USA maintains stable strategic plans, efficiently converting goals.
Although we didn't analyze the defence part yet, we should be able to
know that they excel in defence and counterattacks against the
aggressive Team Canada.\\

For Team Canada:\\
With a more offensive approach, Team Canada persistently seeks goals
regardless of the period or score. Their strategy is pure attack,
considering it the best form of defence. Most players are good at
creating shooting opportunities and exhibit high energy.\\

Although we lack additional plots to analyze defence, we can speculate
on Team Canada's losses. They may lack strong defensive players, relying
too heavily on offence. Additionally, their wide shooting zone suggests
a strategy akin to gegenpressing in soccer, prioritizing attack over
defence, leaving them vulnerable to counterattacks.\\

\newpage

\vspace{2cm}

\begin{enumerate}
\def\labelenumi{\arabic{enumi})}
\setcounter{enumi}{7}
\tightlist
\item
  \textbf{OPTIONAL: PICK 2 OF 4} Discuss briefly any ethical concerns
  like residual disclosure that might arise from the use of your data
  set, possibly in combination with some additional data outside your
  dataset.
\end{enumerate}

My dataset containing detailed event data for women's hockey, it will
normally raise ethical concerns, especially regarding personal privacy,
and information disclosure. Let's combine several points covered in the
lecture and list some ethical considerations.\\

Privacy is the most important topic we need to be aware of when dealing
with statistical activities involving personal information. Thus,
privacy and security considerations are key and mandatory. The dataset
includes information about specific players, teams, and their
performance during games, potentially giving the information to do the
identification of individuals when combined with other events. For
instance, strange player performance or unusual behaviour on the court
could be cross-referenced with external news, compromising their privacy
and potentially harming the players.\\

Moreover, since the dataset encompasses details about players, teams,
and their in-game actions or strategies, it's crucial to ensure whether
the players and teams consent to collecting and using their data for
research purposes. Without proper consent, utilizing the data would
undoubtedly raise ethical concerns regarding respect for individuals'
rights. And also may harm teams and player, and waste their work on
preparing strategies. This is very disrespectful.\\

When conducting statistical activities, it's necessary to consider all
potential risks to the well-being of individuals or specific groups.
Analysis of the data may reveal biases in the game, such as biased
officiating, disparities in player treatment, or discriminatory
practices, which occur frequently across all sports. It can be the
referee on the court, it can be fans in the stadium. Being aware of
these biases is essential to ensure that our analysis does not cause
harm.\\

Statistical activities aimed at benefiting society must be transparent
about data sources, usage, and confidentiality measures. Respect for
data ownership rights and proper attribution to the source are
imperative. Using data without permission or failing to acknowledge
dataset contributions is unacceptable and disrespectful.\\

Conducting analysis or drawing conclusions without considering broader
contexts can lead to misinterpretations. We must be meticulous, striving
for transparency and ethical integrity in our analysis and reporting
endeavors.\\

\end{document}
